\documentclass[11pt]{article}
% -------------------------------------------------
% The standard LaTeX font is Computer Modern. Here, we will load the New
% Century Schoolbook font instead. 
%\usepackage{newcent}
% Other popular options are (name followed by code)
% - Times:
\usepackage{mathptmx}  % **
% - Palatino / Palladio: or \usepackage{tgpagella}
%\usepackage[sc]{mathpazo}  % **
%\linespread{1.05}         % Palladio needs more leading (space between lines)
% - Kp-fonts:
%\usepackage{kpfonts}  % *
% - GFS Didot:
%\usepackage{gfsdidot} % **
% - Utopia:
%\usepackage[utopia]{mathdesign} % *
% - Venturis:
%\usepackage{venturis} % **
% - Libertine:
%\usepackage{libertine} % **
% - Bitstream Vera / Dejavu:
% \usepackage{dejavu} % *

\usepackage[in]{fullpage}
\usepackage[utf8]{inputenc}
\usepackage[numbers]{natbib}
\usepackage{doi}
\usepackage{enumitem}
\usepackage{amsmath}
\usepackage{footnote}
\makesavenoteenv{tabular}
% -------------------------------------------------
\begin{document}

\noindent{\large \bf Truong X. \textsc{Nghiem}}

\paragraph{(a) Professional Preparation} ~

~

\noindent
\begin{tabular}{lllll}
  Hanoi University of Technology & Hanoi, Vietnam & Automatic Control & B.S. & 2003 \\[3pt]
  University of Pennsylvania & Philadelphia, PA & Electrical \& Systems Engineering & Ph.D. & 2012 \\[3pt]
\end{tabular} 

% A list of the individual’s undergraduate and graduate education and postdoctoral training (including location) as indicated below:

% Undergraduate Institution(s)     Location    Major    Degree & Year

% Graduate Institution(s)          Location    Major    Degree & Year

% Postdoctoral Institution(s)      Location    Area     Inclusive Dates (years)

\paragraph{(b) Appointments} ~

~
% A list, in reverse chronological order, of all the individual's academic/professional appointments beginning with the current appointment.

\noindent
\begin{tabular}{llll}
  Northern Arizona University & Flagstaff, AZ    & Assistant Professor            & Jan 2018 -- now      \\[3pt]
  University of Pennsylvania & Philadelphia, PA & Postdoc Researcher & Jun -- Dec 2017 \\[3pt]
  EPFL & Lausanne, Switzerland & Postdoc Scientist & Jan 2015 -- Jun 2017 \\[3pt]
  University of Pennsylvania & Philadelphia, PA & Postdoc Researcher & Oct 2012 -- Dec 2014 \\[3pt]
  University of Pennsylvania & Philadelphia, PA & Research Assistant &  2005 -- 2012 \\[3pt]
  NEC Laboratories America & Princeton, NJ & Summer  Intern & Jun -- Sep 2008 \\[3pt]
  Hanoi University of Technology & Hanoi, Vietnam & Lecturer & 2003 -- 2005
\end{tabular}


\paragraph{(c) Products} ~

~


\noindent{\it (i) Products related to the Proposed Project}

%\nocite{BKL2018:NFM,LBK2018:TAP,BLK2017:VPT,BKL2016:SCAM,BKL2015:FMICS}


\begin{enumerate}
\providecommand{\natexlab}[1]{#1}
\providecommand{\url}[1]{\texttt{#1}}
\expandafter\ifx\csname urlstyle\endcsname\relax
  \providecommand{\doi}[1]{doi: #1}\else
  \providecommand{\doi}{doi: \begingroup \urlstyle{rm}\Url}\fi

\item Sriram Sankaranarayanan, Franjo Ivancic, Aarti Gupta, and Truong X.\ Nghiem.  \newblock {U.S. Patent: System and method for feedback-guided test generation for Cyber-physical Systems using Monte-Carlo}.  \newblock
  Patent number:
  \ifdefined\href
  \href{http://patft.uspto.gov/netacgi/nph-Parser?Sect2=PTO1&Sect2=HITOFF&p=1&u=/netahtml/PTO/search-bool.html&r=1&f=G&l=50&d=PALL&RefSrch=yes&Query=PN/8374840}{US 8,374,840 B2}.
  \else
  US 8,374,840 B2.
  \fi
  \newblock Patent grant date: February 12, 2013.

\item Truong X. Nghiem, Sriram Sankaranarayanan, Georgios Fainekos, Franjo Ivancic, Aarti Gupta, and George J. Pappas.
  \newblock {Monte-Carlo Techniques for Falsification of Temporal Properties of Non-Linear Hybrid Systems}.
  \newblock In \emph{Proceedings of the 13th {{ACM}} International Conference on {{Hybrid}} Systems: Computation and Control ({{HSCC}})}, pages 211-220. Springer, Apr 2010.
  \newblock \doi{10.1145/1755952.1755983}.
      
\item Yash Vardhan Pant, Kartik Mohta, Houssam Abbas, Truong X. Nghiem, Joseph Devietti, and Rahul Mangharam.  \newblock {Co-{{Design}} of {{Anytime Computation}} and {{Robust Control}}}.  \newblock In \emph{Proceedings of the {{IEEE Real}}-{{Time Systems Symposium}} ({{RTSS}})}. IEEE, Dec 2015.
  
\item Truong X. Nghiem, George J. Pappas, Rajeev Alur, and Antoine Girard.
  \newblock {Time-Triggered Implementations of Dynamic Controllers}.
  \newblock In \emph{ACM Transactions in Embedded Computing Systems}, vol.~11, pages 58:1-24.
  Aug 2012.
\end{enumerate}

\noindent{\it (ii) Other Significant Products} ~

\begin{enumerate}
  
  \providecommand{\natexlab}[1]{#1}
  \providecommand{\url}[1]{\texttt{#1}}
  \expandafter\ifx\csname urlstyle\endcsname\relax
  \providecommand{\doi}[1]{doi: #1}\else
  \providecommand{\doi}{doi: \begingroup \urlstyle{rm}\Url}\fi

\item Achin Jain, Truong X. Nghiem, Manfred Morari, and Rahul Mangharam.
  \newblock {Learning and {{Control}} Using {{Gaussian Processes}}: {{Towards}} Bridging Machine Learning and Controls for Physical Systems}.
  \newblock In \emph{International {{Conference}} on {{Cyber}}-{{Physical Systems}} ({{ICCPS}})}.  Apr 2018.
  \newblock \doi{ICCPS.2018.00022}.

\item Truong X. Nghiem, Georgios Stathopoulos, and Colin Jones.
  \newblock {Learning {{Proximal Operators}} with {{Gaussian Processes}}}.
  \newblock In \emph{Annual {{Allerton Conference}} on {{Communication}}, {{Control}}, and {{Computing}}}.
  Oct 2018.

\item Truong X. Nghiem and Colin N. Jones.
  \newblock {Data-Driven Demand Response Modeling and Control of Buildings with {{Gaussian Processes}}}.
  \newblock In \emph{Proceedings of the 2017 {{American Control Conference}} ({{ACC}})}, pages 2919-2924.  May 2017.
  \newblock \doi{10.23919/ACC.2017.7963394}.
  
\item Truong X. Nghiem and Rahul Mangharam.
  \newblock {Scalable {{Scheduling}} of {{Energy Control Systems}}}.
  \newblock In \emph{Proceedings of the {{ACM}} \& {{IEEE International}} Conference on {{Embedded}} Software ({{EMSOFT}})}, pages 137--146.   Oct 2015.
  \newblock \doi{10.1109/EMSOFT.2015.7318269}.

\item Truong X. Nghiem and George J. Pappas.
  \newblock {Receding-Horizon Supervisory Control of Green Buildings}.
  \newblock In \emph{Proceedings of the {{American Control Conference}}}, pages 4416-4421.  Jun 2011.

\end{enumerate}

% A list of: (i) up to five products most closely related to the
% proposed project; and (ii) up to five other significant products,
% whether or not related to the proposed project. Acceptable products
% must be citable and accessible including but not limited to
% publications, data sets, software, patents, and
% copyrights. Unacceptable products are unpublished documents not yet
% submitted for publication, invited lectures, and additional lists of
% products. Only the list of ten will be used in the review of the
% proposal.

% Each product must include full citation information including (where
% applicable and practicable) names of all authors, date of publication
% or release, title, title of enclosing work such as journal or book,
% volume, issue, pages, website and URL or other Persistent Identifier.

% If only publications are included, the heading "Publications" may be
% used for this section of the Biographical Sketch.

\paragraph{(d) Synergistic Activities}

% A list of up to five examples that demonstrate the broader impact of
% the individual’s professional and scholarly activities that focuses on
% the integration and transfer of knowledge as well as its
% creation. Examples could include, among others: innovations in
% teaching and training (e.g., development of curricular materials and
% pedagogical methods); contributions to the science of learning;
% development and/or refinement of research tools; computation
% methodologies, and algorithms for problem-solving; development of
% databases to support research and education; broadening the
% participation of groups underrepresented in STEM; and service to the
% scientific and engineering community outside of the individual’s
% immediate organization.

\begin{enumerate}
\item Member of the IEEE Technical Committee on Cyber-Physical Systems since 2018.
\item Chair of the session ``Verification and Analysis of Hybrid Systems'' at the 2015 {ACM} \& {IEEE} International Conference on Embedded Software ({EMSOFT}).
\item New course ESE~680 ``Digital Twins: Model Based Embedded Systems'' at the University of Pennsylvania in 2017.
\item Presenter on robotics at Flagstaff Coding Camps 2018 for school children, and panelist at the Flagstaff Festival of Science 2018.
\item Research software: main developer of the research software MLE+ for building energy simulation, analysis, optimization and control;  main developer of the research software OpenBuildNet -- a co-simulation platform for large-scale distributed control and simulation of complex multi-agent cyber-physical systems.
\end{enumerate}


% ;;; Local Variables: ***
% ;;; eval: (ispell-change-dictionary "english" nil) ***
% ;;; mode: latex ***
% ;;; eval: (flyspell-buffer) ***
% ;;; End: ***

%\bibliography{bibliography}
%\bibliographystyle{unsrtnat}

\end{document}
