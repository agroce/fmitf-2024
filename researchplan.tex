\section{Research Plan}

%\subsection{From Timed Automata to \framac}
%\label{sec:ta2framac}
%It is notoriously difficult to design correct and secure communication protocols.
One of the most famous examples is the Needham Schroeder Public Key protocol~\cite{NS1978:CACM}.
It took 18 years to discover a flaw in this protocol~\cite{LOW1996:TACAS}, and it was done using formal methods.
Networks of timed automata are formalisms suitable for the formalization of protocols.
They are the basis of model checkers such as \uppaal~\cite{DLL2015:STTT} and \prism~\cite{KNP2011:CAV} that have been used successfully in the verification of network protocols~\cite{ZBW2013:ENTCS,HMJ2006:MASCOTS,HSS2010:NFM,KPK2015:VECOS}.
These tools find issues with abstract formulations of protocols, but cannot help with implementation details that not modeled.

But the {\em implementation} details of a correct protocol also matter as the Heartbleed vulnerability in the OpenSSL implementation shows.
In the case of Heartbleed, the problem was a C runtime error: an access to an invalid memory region, and it was due to an implicit assumption on the input of a function that was actually false.
Combinations of static and dynamic analyses can detect such vulnerabilities~\cite{KKP2015:HVC} because they are not due to complex interactions related to the protocol.

The methodology we envision combines the use of model-checkers such as \uppaal and \prism for the verification of protocols, with frameworks for static and dynamic analysis of C programs, namely \framac and \deepstate, for the verification of the {\em implementations} of protocols.

We consider \framac, and its specification language \acsl pivotal in this approach.
The research challenge here is to translate networks of timed automata into \acsl annotations and C ghost code for enabling the verification of the C code implementing the protocol modeled by the timed automata:

\begin{enumerate}[labelsep=3pt,leftmargin=12pt]
\item The translation of networks of automata into annotations to be used within the \framac code analyzer.
  Previous work co-authored by  co-PI Loulergue on the Contiki~\cite{DGV2004:LCN} lightweight operating system for the Internet of Things showed that various   approaches can be applied to the verification of the same code.
  For checking the correctness of the linked list API of Contiki, it includes the use of ghost arrays~\cite{BKL2018:NFM}.
  Ghost code is a part of a program that is added for the purpose of specification.
  Such code should not interfere with regular code.
  Erasing it should make no observable difference in the program results.
  This approach made it possible to perform most proofs automatically using the \framac/\Wp tool, only a small number of auxiliary lemmas being proved interactively in the \Coq proof assistant
  (later replaced by so-called lemma functions and loop annotations to avoid the use of \Coq~\cite{BLK2019:NFM}).
  This work relied on an elegant segment-based reasoning over the companion array developed for the proof.
  
  This approach, however, is expressed in parts of the \acsl language that cannot be translated to executable C code, i.e. that do not belong to the \eacsl subset.
  In a broader verification context, especially as long as the whole system is not yet formally verified, it is very useful to rely on runtime verification, in particular to test client modules that use the list module.
  A variant of the list module specification~\cite{LBK2018:TAP}  that belongs to the executable subset \eacsl of \acsl can also  be transformed into executable C code.
  A newer approach~\cite{BKL2019:SAC} relies on logic lists: they are part of the \acsl standard library of inductively defined logical data structures.
  In the case of Contiki, a logic list provides a convenient high-level view of the linked list.
  The specifications of all functions is now proved faster and almost automatically.
  All these approaches are based on a predicate, or a combination of a predicate and a logic function, that relates the data structure (linked list), and a representation of it for specification purposes.
  Figure~\ref{fig:representation} shows the logic list representation (left) and array representation (right).
  In the former case, the logic list {\tt ll} represents the linked list from root {\tt bl} to cell {\tt el}.
  In the latter case, the companion array {\tt cArr} ({\tt n} cells from index {\tt index}) represents the linked list from {\tt root} to cell {\tt bound}.


  \begin{figure}
    \lstinputlisting[basicstyle=\scriptsize\ttfamily,multicols=2]{representation.c}
    \caption{Linked List Representation in ACSL}
    \label{fig:representation}
  \end{figure}
  
  We expect several translations of networks of automata to be considered.
  Some may be easier to understand for C programmers not familiar with formal specifications (ghost code).
  Some may be more efficient for deductive verification (logical data structures).
  Some may be more suited for dynamic verification while still being amenable to deductive verification.
  The first approach will rely on ghost code.
  The automata will be translated as C code.
  The states of the model may contain variables: a C structure will be used to represent states.
  Transitions could be modeled as function calls.
  Such translations will be implemented as a \framac plugin, and will be automatic.  However events and states of the automata will need to be associated with respectively specific execution events and memory states of the C programs.
  We expect to experiment manually with this mapping in case studies before enhancing the plugin to provide support for it.
  In particular this means the developer will have to write representation predicates such as those of Figure~\ref{fig:representation} mapping states of the automata model to more concrete and detailed states of the C programs.


\item Although deductive verification about {\em algorithmic complexity} is possible from source code~\cite{WK2009:TYPES,PS2014:SAC,GCP2018:ESOP},
  such a formal approach is not appropriate for this project, essentially because these approaches deal with complexity rather than execution time, and
  there is no precise enough translation from one to the other.

  Time constraints on the transitions will be translated into new \acsl annotations.
  Then we will rely on static analysis tools for worst-case execution time estimation.
  There are recent projects~\cite{MRP2017:WCET} that explore taking advantage of semantics information to improve WCET estimation, as well as preliminary work showing the benefits of such an approach~\cite{BA2014:JRWRTC}.

  A C program with \acsl annotations very often provides a large variety of semantic information including intervals for variable values, and information about loops such as relations between the number of iterations and other variables.
  Exploiting this information will require us to be able to modify the WCET tool.
  This requirement excludes the best WCET estimator, aiT~\cite{FER2004:IPDPS}, a closed source commercial tool.
  Heptane~\cite{HRP2017:WCET} and {\tt OTAWA}~\cite{BCR2010:SEUS} are two actively developed open source projects with software architectures designed to ease extending the tools.
  Heptane is focused on cache analysis while {\tt OTAWA} supports more processor architectures.
  For our case studies, {\tt OTAWA} seems the best alternative for verifying if the bound obtained by WCET estimation on the code indeed satisfies the time constraints obtained for the timed automata.

\item The main part of the code where the specification of the automata will be used should be a kind of event loop.
  However this loop may be incomplete in the sense that it may not consider all the possible events, or even may not be structured as an event loop in the case   most events are handled through interrupts.
  Although \framac does not directly handle concurrency, its \conctoseq~\cite{BKL2016:SCAM} plugin (co-authored by co-PI Loulergue) allows for the analysis of parallel compositions of C programs through program transformation to sequential C programs~\cite{BLK2017:VPT}.
  The main part of the simulating program is a loop that handles control switch among the various threads.
  This loop can be used as the main event loop in a concurrency context.
  
\end{enumerate}



% ;;; Local Variables: ***
% ;;; mode: latex ***
% ;;; eval: (ispell-change-dictionary "english" nil) ***
% ;;; eval: (flyspell-buffer) ***
% ;;; End: ***


The core outcome of this project consists of
  practical methods and a framework for combining formal, static, and dynamic analysis for embedded system software written in C and C++, as well as an open-source software implementation and two real-world case studies.  This section will detail these efforts.

\subsection{DeepState and Automated Test Generation}
\label{sec:framac2deepstate}
%\notetruong{We should update Figure 1 and relate this section to the steps in Figure 1.}
Applying DeepState to real embedded systems requires us to meet many challenges:

\begin{enumerate}[labelsep=3pt,leftmargin=12pt]
\item The \emph{specification} of correctness must be translated into an executable form.  To some extent, the existence of the \eacsl executable subset of \acsl, and libraries for runtime checking of properties satisfies this condition.  DeepState can support any C/C++ executable method of checking for correctness.  However, some executable specifications need to be modified to be efficiently handled when the DeepState back-end is a symbolic execution tool.  DeepState's nature as a test generation tool means that it supports constructs, such as {\tt Minimum}, {\tt Maximum}, and {\tt Pump}, not usually available in executable specifications.  Tailoring \eacsl usage for DeepState therefore requires a custom effort, including extending the semantics of executable specifications and optimizing the implementation for symbolic execution and fuzzing.  Finally, because our domain critically involves timing, we need to implement DeepState handling of (and \eacsl representations for) deadlines, and specification of function-level deadlines including arbitrary, specified, ``runtimes'' for code that operates via simulation rather than real hardware (or in symbolic execution).  Similar, but in some ways even more complex, challenges are posed by the ubiquity of \emph{interrupts} in embedded code, a problem addressed by very little previous work in fuzzing~\cite{song2019periscope}.
\item The \emph{assumptions} that control which tests are considered valid must be translated in the same way; normally, \eacsl simply translates these into further assertions (as pre-conditions to check at runtime), but in DeepState, we need to distinguish between {\tt ASSUME} failures (invalid tests) and {\tt ASSERT} failures (bugs). 
\item The inputs to a function must be translated into code controlling the input values that DeepState provides, including ranges and types.  When input types are simple, this process is straightforward; however, when functions take, e.g., arbitrarily sized arrays, linked lists, or other complex structures, this becomes a problem of constructing a test harness that (1) makes fuzzing and symbolic execution scalable but (2) uses large enough structures to expose subtle bugs.  Moreover, because DeepState supports strategies for input generation, such as forking concrete states for values too complex for symbolic execution using the {\tt Pump} construct, the translation must determine when such strategies are appropriate, and apply them.
\item In many cases, checking a single function may not be an effective way to detect faults; only a sequence of API calls can expose a problem in a system (e.g., that a function produce a state that causes another function to violate an invariant).  \acsl annotations provide enough information for a fully-automated translation to a harness enabling dynamic analysis in the case of proving properties of a single function, but not for groups of functions \footnote{There are some Frama-C plugins related to properties over groups of functions, but none apply to the most general form of state problems easily.}.  Moreover, even in cases where the violation of a specification can, in theory, be discovered without calling multiple functions, the state space may be too large to explore with a fuzzer or symbolic execution tool.  In such cases, exploring only states produced by valid call sequences has two benefits:  first, the space itself may be much smaller, and easier to explore, than the full set of possible input values.  Second, errors in this part of the input space are more important.  Even if a precondition is not sufficiently restrictive to guarantee correct behavior, if the ``bad'' inputs are never, in practice, generated by the functions that modify system state, the fault may not matter.  In cases where constructing a sufficiently exact precondition is difficult for engineers, such ``in-use'' verification may be the only avenue to system assurance.  We propose to let users annotate \emph{sets} of functions to be tested as an API-call-sequence group, extending recent work exploring this concept~\cite{blatter2018static,MetAcsl}.  
  \item Finally, DeepState and, in fact, general-purpose fuzzers such as AFL, have, to date, been exclusively (to our knowledge) used in what might be deemed conventional environments.  As recently noted, ``the tight coupling between hardware and firmware and the diversity found in embedded systems makes it hard to perform dynamic analysis on firmware'' and existing mainstream fuzzing tools offer almost no support to embedded developers for simulation and emulation~\cite{halucinator}.
\end{enumerate}

\begin{figure}[t]
  \hspace{-12pt}
  \begin{subfigure}{0.492\columnwidth}
  {\scriptsize
  \begin{code}
void update\_state(struct state\_t *s, uint64\_t bv) \{
  ASSUME(valid\_state(s));
  ASSUME(valid\_bv(bv));
  ...
\}

void process\_both\_sensor\_readings(struct state\_t *s) \{
  ASSUME(valid\_state(s)); 
  unit64\_t s1\_bv = acquire\_s1(), s2\_bv = acquire\_s2();   
  update\_state(s, s1\_bv);  update\_state(s, s2\_bv);  
\}
  
void process\_one\_sensor\_reading(struct state\_t *s) \{
  ASSUME(valid\_state(s)); 
  unit64\_t s1\_bv = acquire\_s1(); 
  update\_state(s, s1\_bv); 
\}
\end{code}
}
\end{subfigure}
\begin{subfigure}{0.492\columnwidth}
{\scriptsize
\begin{code}
struct state\_t *NewState() \{
  return DeepState\_Malloc(sizeof(struct state\_t));   
\}
    
TEST(SensorReading, UpdateNeverSlow) \{
  struct state\_t *s = NewState();
  DeepState\_Timeout(
    [\&]\{update\_state(s, DeepState\_UInt64());\},
    MAX\_EXPECTED\_UPDATE\_TIME);
\}

TEST(SensorReading, AvoidCrashes) \{
  struct state\_t *s = NewState();
  for(int i = 0; i < TEST\_LENGTH; i++) \{
    OneOf(
        [\&]\{process\_both\_sensor\_readings(s);\},
        [\&]\{process\_one\_sensor\_reading(s);\});
  \}
\} 
\end{code}
}
\end{subfigure}
  \caption{Sensor reading code and DeepState test harness}
  \label{fig:assumption}
  \end{figure}

  These goals require significant advances in three areas of dynamic analysis: first, a complete and principled approach to the problem of handling pre-conditions/assumption semantics, and second, an investigation of how to let fuzzers take advantage of the significant additional structure provided by property-based testing, including such assumptions.  Consider the code in Figure~\ref{fig:assumption}.  This defines two different tests of software that reads sensor values and incorporates them into a system state.  The two tests check two different properties:  {\tt UpdateNeverSlow} ensures that updating the sensor is never too slow.  It is checked, potentially, over \emph{all} valid inputs, not just ones produced by the actual sensor reading code in {\tt acquire\_s1} and {\tt acquire\_s2}.  The second test, {\tt AvoidCrashes} starts the system up in some valid state, and repeatedly either reads both sensors or only sensor one.  There is no explicit property, only the expectation that the system will not crash; tests can be executed using LLVM sanitizers to check for integer overflow and other undefined behavior.  Generating such harnesses automatically from \acsl specifications is a significant challenge, but our research agenda also includes solving problems that would appear even for manual harnesses.  For example, what is the proper semantics of the {\tt ASSUME} in {\tt update\_state}?  It depends on the test.  In {\tt UpdateNeverSlow}, a fuzzer will often generate an input value that violates the (possibly complex) requirements on valid states and sensor readings.  These invalid inputs should not be flagged as bugs (the default behavior of \eacsl), but instead the test should be abandoned without indicating that it failed.  However, in {\tt AvoidCrashes}, since we are not directly generating state values, that is, {\tt update\_state} is not an \emph{entry point} for the test, assumption violations should result in failed tests.  We aim to synthesize code to make assumptions automatically take on the proper semantics during test execution (including symbolic execution using constraint solvers).

  This point about preconditions/{\tt ASSUME} brings up a second point.  Preconditions, when they have an {\tt ASSUME} semantics, are fundamentally different than other branches in code.  A fuzzer will attempt to explore the behavior of branches in {\tt valid\_state} and {\tt valid\_bv} just as it explores branches in {\tt update\_state} or {\tt acquire}.  However, it is often possible to enumerate a vast number of paths that differentiate only invalid inputs, and so produce very little real testing.  A classic example is ``testing'' a file system by producing a huge variety of unmountable file system images, rather than actually executing POSIX operations~\cite{CFV08,AMAI}.  DeepState knows which branches are pre-conditions, and so can help avoid this problem.  In some fuzzers, this means prioritizing inputs to mutate based on whether they execute any code other than validity checks; but in fuzzers, such as Angora~\cite{angora} and Eclipser~\cite{eclipser}, that use lightweight constraint-solving to cover branches, the process can be more sophisticated.  We have begun discussions with the Eclipser team, and they confirm that identifying precondition code and devising suitable heuristics to handle it (e.g., never solve for a negation of a passed check) should improve performance.  Fuzzing of individual functions or sets of functions is a highly promising area: most fuzzing is applied at the whole-program level, where input generation can simply be too hard.  By focusing on a middle-ground between unit testing and whole-program fuzzing--using fuzzer technology to drive property-driven testing--the problem is made tractable.  Prioritizing paths that include more than just input validation is an explicit goal of, e.g., AFLFast~\cite{aflfast}, but it must work with an implicit definition based on path frequencies, while we have access to ground truth.  Given the complexity of state validity checks, there may be hard-to-reach---but uninteresting---ways to create invalid input; AFLFast will \emph{prioritize} such paths, while we will (correctly) avoid them.

  However, it is possible to be more aggressive with preconditions that flow directly from the test harness to a function.  Namely, in a large number of cases, it is possible to \emph{actually produce values satisfying a precondition from a fuzzer-chosen value} rather than simply abort the run.  This does not violate the semantics of the SUT; it merely transforms one arbitrary input into another, with a fixed mapping so that the fuzzer can still learn from the pattern of inputs, \emph{as transformed}.  Consider the simple case of ranges.  If a function begins with {\tt ASSUME ((input > 10) and (input < 256))} where {\tt input} is an integer parameter to the function, and this assumption takes place before any assignments to {\tt input} in the function, then in the case where the fuzzer directly generats a value for {\tt input}, we can replace the assumption with the code:

  {\tt if ((input <= 10) || (input >= 256)) input = (abs(input) \% 245) + 11}

  Rather than testing if {\tt input} is in a range, this simply maps values that violate the assumption into valid values, that could have been provided by the fuzzer.  In such simple cases, of course, the developer of the test harness could have written {\tt DeepState\_IntInRange(11, 245))}but when the values in an assumption are dynamically computed inside the called function and/or involve state values not visible to the test harness, or are simple more complex constraints than a simple range check, this is often impractical and sometimes almost impossible.  We propose to provide forms of {\tt ASSUME} that enable automatic mapping of inputs, without developer action (other than using our variants of ASSUME), using a mix of guaranteed transformations such as in the range case and ``search-based'' approaches that find a nearby value of an input that satisfies an assumption (particularly useful in case of, e.g., parity checks).

As far as we are aware, the problem of mapping inputs (rather than producing inputs) to satisfy a predicate has not been explored in the literature.  Note that a ``mapping'' may in the most general case be a \emph{search}: while also amenable to a solution like that above, a parity check can be satisfied simply by incrementing a fuzzer input byte until it satisfies a check.  This most-general approach may be useful in some cases, such as complex checksums with a small size, where a complete solution cannot be produced.  On average the time to produce a 16 bit checksum by brute force search from random bytes, for example, may be an acceptable overhead to fuzzing.

 We note that this approach to preconditions is not limited to DeepState and/or embedded systems; a manual transformation of this sort is given as essential advice for users of the Echidna smart-contract fuzzer \cite{echidna-advice}.  However, while Trail of Bits even provides an API for the simple range case, the complexity of correctly using the shift from assumption to mapping is such that it is seldom done for more complex cases.
  
  This effort also connects to a second fuzzing research thrust: making specification elements that do not correspond to simple code coverage visible to a fuzzer.  In this example, consider the {\tt DeepState\_Timeout} check (note that this itself is functionality we will develop as part of handing timing constraints in \framac and DeepState).  Unless we break down the timing analysis explicitly using a set of conditional branches, coverage-driven fuzzers cannot distinguish an execution that is very slow (close to violating the constraint) from one that has the minimum execution time possible.  We propose to make timing of such specified events visible to a fuzzer, by modifying coverage bit-vectors to incorporate bucketing of execution time.  Once we add such novel coverage measures, and introduce distinctions between coverage classes (as with preconditions), we will research how to balance competing priorities in more complex notions of coverage.  In addition to implicit execution properties such as timing, this can apply to coverage of data structures, for fuzzing data-driven code such as machine-learning algorithms, where much behavior is implicit---e.g., the route taken through a forest of decision trees.  In general we aim to extend the work~\cite{aflfast,lemieux2018fairfuzz,vuzzer,zhao2019send,aschermann2019redqueen}, that prioritizes certain program paths in an intelligent way, by exploiting our extended \acsl/\eacsl.

Finally, these elements must be tied to the problem of applying fuzzing and related methods in embedded-relevant execution environments.  We plan to investigate multiple potential solutions, initially focusing on integrating fuzzing with the HALucinator tool~\cite{halucinator} for virtualizing firmware via the Hardware Abstraction Layer, which is likely to add emulator-specific notions of coverage and path relevance.

% Additionally, in some cases, checking a single function may not be an effective way to detect faults; only a sequence of API calls can expose a problem in a system (e.g., that a function produce a state that causes another function to violate an invariant).  \acsl annotations provide enough information for a fully-automated translation to a harness enabling dynamic analysis in the case of proving properties of a single function, but this is no longer true for groups of functions.  Moreover, even in cases where the violation of a specification can, in theory, be discovered without calling multiple functions, the state space described by the precondition for a function may be too large to explore with a fuzzer or symbolic execution tool.  In such cases, exploring the space described by valid calls of other functions has two benefits:  first, the space described by a sequence of calls may be much smaller, and easier to explore, than the full set of possible input values to a function.  Second, errors in this portion of the input space are more clearly realistic scenarios.  Even if a precondition is not sufficiently restrictive to guarantee correct behavior, if the ``bad'' inputs are never, in practice, generated by the functions that modify system state, the fault may never appear in practice.  In cases where constructing a sufficiently exact precondition is difficult for engineers, such ``in-use'' verification may be the only avenue to system assurance; proof is impossible without a restrictive enough precondition, and dynamic methods may scale very poorly to, e.g., a large unstructured byte buffer such as a hash table.

% In principle, of course, users can write a new function (a kind of ``ghost function'' not really executed---in practice, a test harness) expressing the desired mix of API calls that preserve an invariant; however, this is a serious burden on a user, and users are likely to make errors in this task~\cite{CFV08,AMAI,scriptstospecs,groce2015verified,groce2018verified}; we instead propose to let users annotate (in an extension of \acsl) sets of functions to be tested as an API-call-sequence group.  E.g., annotating a set of file system functions ({\tt mkdir}, {\tt rmdir}, {\tt readdir}, etc.) as such a group could allow the automatic generation of a DeepState harness that checks for cases where a sequence of valid function calls can violate a precondition or cause a fault despite preconditions being satisfied.


%%% Local Variables:
%%% mode: latex
%%% TeX-master: "main"
%%% End:


\subsection{Other DeepState Extensions}

\paragraph{Bounded Model Checking:}
While automated test generation by fuzzing or binary-level symbolic execution can be highly effective as a means for finding bugs in code, other approaches are also needed to handle the kinds of code especially common in embedded contexts.  In particular, embedded software often includes a large number of functions that perform complex low-level bit operations, especially for interacting with hardware and ``parsing'' network packets (from traditional wireless or RF-derived signals).  Fuzzing or binary symbolic analysis often has trouble  finding exact bit-values; it is well known that, e.g., inverting even non-cryptographic hashes is hard.  Translation to SAT or SMT, however, often easily handles such problems.

CBMC, the C Bounded Model Checker~\cite{cbmcp} is a well-known tool that analyzes C programs using a translation to SAT or SMT queries based on a bounded unrolling of loops. CBMC is an actively developed project, and has been used extensively in real-world development for years, including in automotive/embedded code development at Bosch and General Electric~\cite{tiemeyer2019crest}, in analysis of Amazon Web Services infrastructure~\cite{awsmodel}, and in the analysis of flight software systems at NASA's Jet Propulsion Laboratory~\cite{AMAI}.  Using CBMC requires writing custom test harnesses using CBMC's API for expressing nondeterminism, and running the tool with a specified bound on loop executions, in addition to other complex configuration options.

We propose to allow CBMC to be used as a backend for verification by DeepState, with a seamless interface, just as DeepState currently supports symbolic analysis engines such as angr and Manticore.  It is notoriously hard to guess when a SAT/SMT based approach to code analysis will work well and when it will fail to scale; using a DeepState harness will allow users to try CBMC at ``no cost.''

Moreover, because choosing loop unwinding bounds imposes a serious burden on embedded engineers, we will investigate their automatic determinations.  One approach is to instrument fuzzer or symbolic-execution engine generated tests to record iterations of loops, and then use the maximum bound observed.  Additionally, for small functions (the most likely targets for DeepState-CBMC: complex but compact bit-manipulation code), the mutation-based approach proposed by Groce et. al~\cite{groce2018verified} may work.  Finally, in some cases CBMC may be able to find interesting bugs for cases where the loop unrollings are limited, but cannot scale to larger depth limits.  Using the same instrumentation that we use to estimate loop bounds, we will use the ability to guide fuzzers by alternative ``coverage'' to focus fuzzer runs on executions with more loop iterations than the bound explored by CBMC.  This will offer engineers a true partnership between verification methods. 

\paragraph{Explicit-State Model Checking:}
Just as some functions are best analyed using bounded model checking, some dynamic analysis problems are best handled by explicit-state model checking that actually executes C code, like a fuzzer, but with the capability to store states and backtrack, in order to exhaustively explore a state space, using either actual comparison of stored states or comparison of abstractions of states to guide exploration.  This approach is particularly attractive for exploring sequences of API calls; this kind of test generation was used in efforts that uncovered dozens of errors in file systems at NASA/JPL~\cite{AMAI}.

The SPIN model checker~\cite{SPIN} offers execution of C code with backtracking~\cite{ModelDriven,ModelCode}.  DeepState's {\tt OneOf} construct has a semantics that can be matched with the SPIN nondeterministic choice, which in part inspired the DeepState construct~\cite{WODA08,WODACommon}.  However, integrating SPIN as a back-end for DeepState is even more challenging than integrating CBMC.  With CBMC, the mapping from DeepState to CBMC semantics may be performed by changing included headers so that CBMC-specific constructs have differing implementations (but not semantics); SPIN however executes C code in the context of a PROMELA model, which requires rewriting a DeepState model to embed test choices inside SPIN's constructs.  This also means ``lifting'' DeepState API calls to the PROMELA level outside the C code, and bridging between nondeterminism visible to SPIN and determinism within C code; PI Groce's previous work~\cite{WODA08} can serve as a foundation.  A more fundamental problem is that while CBMC and DeepState can share a semantics for, e.g., {\tt DeepState\_Int64()}, a PROMELA model with a branching factor of, e.g., $2^{64}$ will not work.  Solutions range from using results from fuzzing to choose a limited range, to translating ``flat'' bit-value selection into a sequence of choices with a larger range but bias towards certain values, to using SPIN to control a seed and deterministically choosing random values~\cite{WODA08}, a hybrid approach.  

\paragraph{Timed Automata Model Skeleton Generation:}
As noted above, one of our core assumptions is that timed automata can
model the underlying protocols in many embedded systems.  However,
writing timed automata models using \uppaal~\cite{uppaal} and
\prism~\cite{KNP2011:CAV} is at present a skill only a small number of
embedded engineers have mastered.  In order to encourage more
engineers to make use of these powerful formalisms, we propose 1) to
enable DeepState to generate \emph{traces} of the annotations related
to timing that are covered during a run and 2) to build a tool to
combine and reconcile these traces into a skeleton model for \uppaal~\cite{uppaal} or
\prism~\cite{KNP2011:CAV} (as has been done to some extent for Java~\cite{liva2017extracting}).  The structure of code (function locations
of DeepState annotations) will be used to form the structure of the
model.  Additional annotations for, e.g. probabilities, may need to be
added if not present in the code annotations, though DeepState already
has a primitive support for expressing probabilities that we plan to extend.

% \paragraph{Other Tools:}  Galois Inc. has expressed strong interest in
% an (unpaid) ongoing research collaboration to integrate their C and
% C++ relevant tools into DeepState as well, to support our general vision.

\subsection{Pre-Trained Models to Provide Annotation Assistance}

As noted in the problem statement, while developers are almost all at
least familiar with writing unit tests, extending this experience to
annotating embedded code and to writing unit tests \emph{not limited
  to single concrete values} is a challenge.  While using large
language models (LLMs) to generate or complete code in critical
embedded domains is risky, such models do provide a way to support
developers in testing and verifying their code.

Modern LLMs have been pretrained on millions of source code tokens, thus these models have a strong understanding of code semantics 
and behavior. This can be seen with CodeBERT~\cite{codebert} embeddings outperforming prior work such as Code2Vec~\cite{code2vec} 
and TF-IDF~\cite{tfidf}, with the CodeBERT vectors capturing some
notion of semantic similarity rather than pure token match (as in techniques 
such as TF-IDF).  These training bases contain a large number of
tests, with many signals including naming and structure distinguishing
the test code within these corpuses, making
models ``aware'' of the tightly constrained nature of test code.
Thus one can generate entire test suites with a powerful model such as ChatGPT~\cite{gpttestgen, siddiq2023empirical}. 
Even smaller models, trained with this relationship~\cite{catlm, starcoder} show significant promise, generating tests that compile, execute successfully, and 
increase coverage.  The context of existing embedded systems code gives LLMs more to work with,
and the context of existing tests for production code may give LLMs
even more to work with: LLMs have already been used to augment existing tests
with (better) oracles~\cite{OracleGEN}.  This is a particulary promising
application, since the tendency of oracle quality to lag behind code coverage
is an enduring and under-addressed problem in software
testing~\cite{MindGap}, and one of the problems our proposal targets.

Again, the task most similar to our approach is that of
writing property-based tests or fuzz-drivers~\cite{goldstein2022some},
which, while more developer friendly than any other route to formal
methods application we know of, is sufficiently novel to discourage
developers in their first uses.  Generalizing oracle
generation from specific to parameterized unit tests is a potential
long-term solution to this problem, and there is already work underway
to generate fuzz drivers using LLMs~\cite{zhang2023understanding},
including by Google's OSS-Fuzz team, one of the most important
industrial fuzzing efforts~\cite{ossfuzzllm}.  PI Groce has recent experience
using CodeBERT for a different, but related, engineering task
(predicting the impact of code changes on tests, in particular for
mutation testing ~\cite{ContextPMT}) and will carry the lessons learned in that work to
this domain.

We therefore aim to pre-train models (probably based on CodeBERT) both
for generation of harnesses and code annotations and for use in
autocompletion of developer-written annotations.  This specialized
version of the general code-generation task is easier to train and
evaluate, due to the structural and semantic restrictions of test code
and specification annotations.  Evaluation of models will be performed
using both the usual machine-learning methods as in the above
discussed recent framing work and using the same
developer/task oriented methods as in our other case-study based work.

\subsection{Dissemination of Software}
\label{sec:software}

The proposed methods and framework will be incorporated into a usable
open source tool.
While individual software components will be developed and evaluated
in tandem with the research efforts described above, we dedicate a
task (Task T3 in the workplan given in the separate Collaboration Plan) for consolidating all
these components in a practical, user-friendly software tool since
adoption by real-world users is a primary scientific goal of this project.
To maximize impact, the software will be made publicly available % on GitHub
using the MIT open source software license.
We are committed to making software open source as demonstrated by previously released % research
code \cite{mleplussoftware, bernalMLEToolIntegrated2012a, openbuildnetsoftware, nghiemOpenBuildNetFrameworkDistributed2016}.


\subsection{Case Studies} % for Ecological Monitoring and Control}
\label{sec:case-study}

The above briefly introduces a number of problems that we know in advance must
be dealt with in order to enable a pathway for combining formal,
static, and dynamic analysis.    At heart, however, we aim to allow
case studies to prioritize our efforts, and
are certain that other challenges will arise during these efforts.
The studies informing this research are the embedded software of %wireless sensor nodes used in the Southwest Experimental Garden Array (SEGA)~\cite{ClaEtAl11,GhoEtAl2014,BelEtAl2015} and of 
wireless sensor nodes and mobile robots in the Distributed Sensing \& Computing Over Sparse Environments (DISCOVER) Platform.

\subsubsection{Overview of the DISCOVER Platform}
\label{sec:cast-study:discover}

DISCOVER %(Distributed Sensing and Computing Over Sparse Environments)
is a cyber-infrastructure testbed for remote, rural, and sparsely populated areas.
The project is funded by NSF and led by NAU, whose team includes co-PI Nghiem (co-PI of DISCOVER).
DISCOVER consists of a fabric of highly configurable Internet-of-Things (IoT) sensor nodes, autonomous and highly capable terrestrial robots and drones, and a heterogeneous wireless network.
DISCOVER sites will be located at the campuses of NAU, Navajo Technical University, and Clemsom University, as well as several remote sites.
The platform will enable focused research in many domains, including data science and machine learning, heterogeneous networked services, distributed computing and AI, control, autonomous robots, and in-network computation, among many others.
%
We will use DISCOVER for the case studies in this project.

% \subsubsection{Communication Protocol for Wireless Sensor Nodes in SEGA}
% \label{sec:sega-case-study}

% \paragraph{Overview:}

% SEGA is a large collection of operational wireless sensor/actuator networks for monitoring and control of ecological systems, located at 17 sites in the states of Arizona and California.
% Currently, SEGA consists of 138 wireless nodes and is planned to expand to a total of 154 nodes at 21 sites in the coming years.
% As a genetics-based climate change research platform, SEGA allows scientists to quantify the ecological and evolutionary responses of species to changing climate conditions.
% Multiple long-term and large-scale scientific experiments are conducted at SEGA sites.
% The SEGA project was led by NAU, and the group of co-PI Flikkema developed and are maintaining the wireless nodes, including their embedded software.
% %
% The SEGA nodes use a multi-processor architecture, in which a central processor provides OS-level services %, including scheduling and dispatch of tasks, storage, and a message-passing interface for wireless networking.
% while plug-in satellite processors handle transducer sampling, actuation, and related computational tasks.
% In addition to allowing true parallelism, this architecture enables hardware-level improvements in energy efficiency, since each satellite can be optimized for its specific task.
% % The current implementation of the architecture emphasizes energy efficiency~\cite{FliSENSORS2010,FliICC2011}.
% % For example, all satellite processors are power-gated via central processor control; ensuring that satellite processors are depowered prevents satellite sleep-mode energy leakage.
% % The power subsystem provides multiple power buses at different voltages, including an optically-isolated high-power bus for actuation.
% % A variety of energy supplies are also supported, including battery-backed photo-voltaic sources~\cite{FliEtAl12,KnaFli17}.
% %
% The nodes synchronously interact with neighbors in a multi-hop, self-organizing/healing network, implemented by a custom communication protocol designed by the group of co-PI Flikkema. %; synchronization is implemented as scheduled rendezvous in time slots; slot boundaries are managed by a lightweight global time synchronization protocol that is integrated with low-level communication synchronization.
% The nodes use a custom time-triggered RTOS tightly integrated with a time/frequency-hopped PHY/MAC protocol.
% This approach %, implemented using a time-triggered architecture on a custom RTOS,
% minimizes communication energy cost.
% \paragraph{Challenges:}

% Because timing is critical and is determined by the embedded system hardware and software, most testing has occurred at the network level, with extensive in-lab testing with small networks and instrumented field tests.
% However, it has been found in long-term deployments % (at dozens of field sites over years of operation)
% that occasionally the networking fails and nodes become isolated---we think due to a complex set of subtle bugs rooted in different levels of timing abstraction.
% When such a failure occurs, it often spreads from one node to others, causing nodes to seek to rejoin and expend high levels of energy for radio operation and eventually deplete their energy sources.
% Eventually, subnets, or sometimes the entire site, are disabled and humans must visit the site to reboot it.
% Such failures could %cause scientific data to be lost or invalidated and, even worse, could
% affect %damage
% or even destroy (e.g., via over-watering), long-running scientific experiments.
% %
% We aim to use SEGA (in particular the protocol in question and its implementation) as our case study.
% This will enable us to apply our approach in a practical setting, and ensure that what we produce is actually usable by engineers of real systems.
% %
% SEGA is an ideal case study for several reasons.
% First, the above mentioned network problem enables exploring how to design, prove, and test time-critical systems in a way that does no harm: human life is not affected in this application, and data is not lost since all sensed information is logged as a local back-up.
% On the other hand, reliable operation is important. %, and failure costly. %, because access to manually fix problems can be problematic even with current installations, and in the future this problem will only grow.
% Finally, this application uses common data structures for task control blocks, and the operating system at each node schedules and dispatches both periodic and pseudo-randomly scheduled tasks.



% \paragraph{Plan:}
% %%%%%%%%% How will we carry out this case study?

% Following our proposed workflow, we will first annotate the implementation with specifications of correctness properties.  We may model the protocol itself as a timed automaton in \uppaal or \prism, in order to ensure that there is not a subtle flaw in the protocol itself, and to model our expectations of behavior in the real system (and to better understand needed specifications).
% Either of these steps may expose the source of the mysterious networking failures.
% We will use DeepState, driven by harnesses automatically generated by our tools, to generate tests of the implementation components in question, using fuzzing at first, followed by CBMC and SPIN model checking once prototype back-ends are available.
% For the purpose of verification and testing, the uncertainty inherent in the physical environment and in wireless communication will be modeled by probabilistic timed automata.
% %DeepState testing may expose faults that are not part of the specification.
% % TRUONG: I commented out the following sentences because they are general (why we want to complement \framac with DeepState), therefore they should be included in the general discussion of the workflow.
% %
% %For instance, using libFuzzer with DeepState we can use LLVM's Undefined Behavior sanitized to catch some classes of undefined behavior that \framac does not take into account.  Furthermore, \framac's ability to prove properties about interactions of multiple functions operating in arbitrary sequence is often limited; such proofs are notoriously hard to construct in general.  DeepState allows us to hope to detect faults when we cannot prove correctness.  DeepState's ability to use symbolic execution as a back-end will be most useful for verifying single functions that are hard to verify with \framac, while state-of-the-art fuzzers will be most useful for sets of functions, or cases where symbolic execution fails to scale.
% %
% The above workflow will be conducted by an Embedded System Engineering student, who is familiar with the SEGA IoT system but does not have expertise in software verification and testing, using the software tools developed in this project.
% Feedback from the engineer in this case study will inform us how to develop and improve the theory and tools for practical usage.


\subsubsection{Case Study 1: Embedded Software of Wireless Sensor Nodes in DISCOVER}
\label{sec:case-study:DISCOVER}

\paragraph{Overview and challenges:}
As a community research platform, DISCOVER will allow users, who are researchers in relevant field domains, to develop software code and experiments for the DISCOVER stationary nodes (i.e., sensor nodes) and mobile nodes (i.e., terrestrial robots and drones).
A critical step of the process supported by DISCOVER is automatic verification and testing of the embedded code submitted by users for live experiments.
We expect that submitted code is developed by researchers who are not trained in computer science and who do not usually apply best practices in software engineering.
We also expect that submitted code will have a wide spectrum of code quality and may be malicious, either by chance or intentionally.
Automatic functional testing of user code will therefore be of critical importance for the operation and sustainability of DISCOVER.
Another challenge is the fact that the code to be tested and verified is for embedded systems that have highly complex physical dynamics %(in the case of robots) 
and interactions with the physical world and with other physical systems, and are constrained by limited computing power and energy.
In addition, wireless communication in a WSN is often subject to frequent packet drops and other failures.
Such physical aspects cannot be easily described in code for the purpose of software verification and testing.
Therefore, sophisticated software-based and hardware-in-the-loop simulations %, including embedded hardware emulators and robot simulators, 
are necessary, for which several options will be developed by the DISCOVER team.

\paragraph{Plan:}
Following our proposed workflow, we will first annotate the implementation with specifications of correctness properties.
%We may model the protocol itself as a timed automaton in \uppaal or \prism, in order to ensure that there is not a subtle flaw in the protocol itself, and to model our expectations of behavior in the real system (and to better understand needed specifications).
% Either of these steps may expose the source of the mysterious networking failures.
We will use DeepState, driven by harnesses automatically generated by our tools, to generate tests of the implementation components in question, using fuzzing at first, followed by CBMC and SPIN model checking once prototype back-ends are available.
For the purpose of verification and testing, the uncertainty inherent in the physical environment and in wireless communication will be modeled by probabilistic timed automata.
%DeepState testing may expose faults that are not part of the specification.
% TRUONG: I commented out the following sentences because they are general (why we want to complement \framac with DeepState), therefore they should be included in the general discussion of the workflow.
%
%For instance, using libFuzzer with DeepState we can use LLVM's Undefined Behavior sanitized to catch some classes of undefined behavior that \framac does not take into account.  Furthermore, \framac's ability to prove properties about interactions of multiple functions operating in arbitrary sequence is often limited; such proofs are notoriously hard to construct in general.  DeepState allows us to hope to detect faults when we cannot prove correctness.  DeepState's ability to use symbolic execution as a back-end will be most useful for verifying single functions that are hard to verify with \framac, while state-of-the-art fuzzers will be most useful for sets of functions, or cases where symbolic execution fails to scale.
%
The above workflow will be conducted by an Embedded System Engineering student, who is familiar with the DISCOVER wireless sensor nodes but does not have expertise in software verification and testing, using the software tools developed in this project.
Feedback from the engineer in this case study will inform us how to develop and improve the theory and tools for practical usage.


%%% Local Variables:
%%% mode: latex
%%% TeX-master: "main"
%%% End:


\subsubsection{Case Study 2: Distributed Coordination in Multi-Robot Systems}
\label{sec:case-study-robots}

%%%%%%%%%%%%%%%%%%%% Intro to the application

\paragraph{Overview:}

Coordinated operation of multiple autonomous robots %(multi-robot systems)
has many important real-world applications~\cite{multirobot2005,multirobotsurvey2013}, e.g., in rescue, security, or disaster response missions. %, several autonomous aerial robots can coordinate to survey an area, monitor target objects, % or activities,
%and guide ground robots. % or vehicles.
In such applications, each robot is autonomous but has the capability to coordinate efficiently and safely with other robots to complete a shared mission, often in a distributed manner. % without any central coordinator.
Such coordination is essential in real-world applications where the environment is constantly and unexpectedly changing.
One of the most critical challenges of this application is to guarantee the safety of a coordination plan, which is typically implemented in C code on the embedded computers of the robots and usually involves wireless inter-robot communication, sensing, and actuation.
The terrestrial robots and aerial drones of the DISCOVER platform will be used for this case study.


\paragraph{Challenge:}
Validation of a distributed coordination method for a multi-robot system is currently performed using a mix of theoretical proof (for limited settings), extensive computer-based simulations, simulation-based falsification techniques, and real-world tests with robots.
Even when a method is validated by proofs and/or simulations, it often fails in real tests due to discrepancies between models and reality/implementation.
The methods and tools proposed in this project will help control and robotics researchers, who usually do not have expertise in software verification and testing, overcome this challenge.


\paragraph{Plan:}
First, we will model a coordination plan %/algorithm
  for multiple robots as a (potentially very complex) network of timed automata.
  Performance specifications will be expressed in temporal logics, e.g., the Signal Temporal Logic (STL)~\cite{donze2010robust}, and checked against the model using verification and testing tools such as \uppaal or S-TaLiRo~\cite{annpureddy2011s}.
  While we do not expect actual user code to be accompanied by formal models, in our case study, this step ensures that the original coordination plan has no subtle flaws, and helps us determine properties that need formulation at the implementation level.
  An implementation of the algorithm in C code, distributed among the robots, will be developed by a robotics/control student.
  The implementation will be annotated with a specification in our extended \acsl/\eacsl.
  We will then use DeepState harnesses to generate tests of the implementation components using fuzzing, symbolic execution, and both bounded SAT/SMT based and explicit-state model checking.
  Finally, we will determine if a timed automata skeleton extracted from the implementation code corresponds to and would help create a full specification such as we developed before beginning implementation.
  % This case study will be conducted by a robotics/control graduate student in the ICONS Lab, using the software tools developed in this project.
  The very different nature and complexity of this study, compared to stationary sensor nodes, will ensure that our methods and tools work in a variety of kinds of real systems.
  % Given the different nature and complexity of this application compared to the SEGA study, the feedback will be much valuable for the development and improvement of the proposed methods and tools for practical usages in a wide spectrum of real systems.
  To overcome the challenge stemming from the complex physical dynamics and interactions of the robots, we will utilize a sophisticated robot simulation environment, based on the Robot Operating System (ROS) \cite{ROS}, with a rich set of predefined scenarios, developed by the DISCOVER team (specifically by the group of co-PI Nghiem).
  An interface between the robot simulation software and the tools developed in this project will be created to enable seamless verification and testing of the robotic code.



%%% Local Variables:
%%% mode: latex
%%% TeX-master: "main"
%%% End:


\subsection{Work Plan}

The detailed work plan and evaluation methods for our proposal are
described, including timelines, in the collaboration plan for this proposal.

%\label{sec:workplan}
%\begin{wrapfigure}[11]{r}{.35\textwidth}
%\begin{figure}[!tp]
%  \centering
  \resizebox{.35\textwidth}{!}{%
  \begin{ganttchart}[%Specs
    hgrid style/.style={black, dotted},
    vgrid, %={*2{black,dotted}, *1{black, dashed},
      %*2{black,dotted}, *1{black, dashed},
      %*2{black,dotted}, *1{black, dashed},
      %*2{black,dotted}, *1{black, solid}},
    x unit=3mm,
    y unit chart=5mm,
    y unit title=5mm,
    %time slot format=isodate,
    title height=1,
    milestone label font=\footnotesize,
    group label font=\bfseries\footnotesize,
    title label font=\bfseries\footnotesize,
    link/.style={->, thick},
    %bar/.style={fill=blue},
    %bar height=0.7,
    %group right shift=0,
    %group top shift=0.7,
    %group height=.3,
    %group peaks width={0.2},
    %inline
    ]{1}{36}
    % labels
    % \gantttitle{A two-years project}{24}\\  % title 1 
    \gantttitle[]{Year 1}{12}                 % title 1
    \gantttitle[]{Year 2}{12}
    \gantttitle[]{Year 3}{12} \\
    \gantttitle{Q1}{3}                      % title 3
    \gantttitle{Q2}{3}
    \gantttitle{Q3}{3}
    \gantttitle{Q4}{3}
    \gantttitle{Q1}{3}
    \gantttitle{Q2}{3}
    \gantttitle{Q3}{3}
    \gantttitle{Q4}{3}
    \gantttitle{Q1}{3}
    \gantttitle{Q2}{3}
    \gantttitle{Q3}{3} 
    \gantttitle{Q4}{3}\\    

    % \ganttgroup[inline=false]{Group 1}{1}{5}\\ 
    % \ganttbar[progress=10,inline=false]{Planning}{1}{4}\\
    % \ganttmilestone[inline=false]{Milestone 1}{9} \\

    % \ganttgroup[inline=false]{Group 2}{6}{12} \\ 
    % \ganttbar[progress=2,inline=false]{test1}{10}{19} \\
    % \ganttmilestone[inline=false]{Milestone 2}{17} \\
    % \ganttbar[progress=5,inline=false]{test2}{11}{20} \\
    % \ganttmilestone[inline=false]{Milestone 3}{22} \\       

    % \ganttgroup[inline=false]{Group 3}{13}{24} \\ 
    % \ganttbar[progress=90,inline=false]{Task A}{13}{15} \\ 
    % \ganttbar[progress=50,inline=false, bar progress label node/.append style={below left= 10pt and 7pt}]{Task B}{13}{24} \\ \\
    % \ganttbar[progress=30,inline=false]{Task C}{15}{16}\\ 
    % \ganttbar[progress=70,inline=false]{Task D}{18}{20} \\ 

    \ganttgroup[
        group/.append style={fill=blue}
    ]{WP1}{1}{36}\\ [grid]
    \ganttbluebar[
        name=T11
    ]{T1.1}{1}{12}\\ [grid]
    \ganttbluebar[
        name=T12
    ]{T1.2}{13}{36}\\ [grid]
    % \ganttlinkedbluebar{}{2014-10-7}{2014-10-10}
    % \ganttlinkedbluebar{}{2014-10-14}{2014-10-15}
    % \ganttlinkedbluebar{}{2014-10-17}{2014-10-17}
    % \ganttlinkedbluebar[name=FMEend]{}{2014-10-21}{2014-10-24}
    % \ganttlinkedbluebar{}{2014-10-28}{2014-10-31}\\ [grid]
    % \ganttbluebar[name=Manual]{Manual}{2014-10-30}{2014-10-31}
    % \ganttlinkedbluebar{}{2014-11-4}{2014-11-7} \ganttnewline[thick, black]

    \ganttgroup[
        group/.append style={fill=blue}
    ]{WP2}{1}{36}\\ [grid]
    \ganttbluebar[
        name=T21
    ]{T2.1}{1}{12}\\ [grid]
    \ganttbluebar[
        name=T22
    ]{T2.2}{13}{36}\\ [grid]

    \ganttgroup[
        group/.append style={fill=blue}
    ]{WP3}{1}{36}\\ [grid]
    \ganttbluebar[
        name=T3
    ]{T3}{1}{36}\\ [grid]
    \ganttbluebar[
        name=T41
    ]{T4.1}{1}{12}\\ [grid]
    \ganttbluebar[
        name=T42
    ]{T4.2}{13}{36}\\ [grid]
    \ganttbluebar[
        name=T51
    ]{T5.1}{7}{12}\\ [grid]        
    \ganttbluebar[
        name=T52
    ]{T5.2}{13}{36}\\ [grid]
    \ganttbluebar[
        name=T53
    ]{T5.3}{25}{36}    
    % %Implementing links
    % \ganttlink[link mid=0.75]{Documentation}{FME}
    % \ganttlink{FMETutorial}{FME}
  \end{ganttchart}}%
\caption{Project schedule.}%
\label{fig:project-schedule}%
% \end{figure}
%\vspace{-0.4in}
\end{wrapfigure}

The project will be organized into two phases, described by work
packages.  In the first phase, T4.1 will be conducted along with and inform T1.1 (see Figure~\ref{fig:project-schedule}).
In the second phase, the focus will be on the application of tools in T1.2 in tandem with T2.
Tasks related to the case studies %(tasks T4.1, and T4.2)
will help refine the developed tools especially in the final phases of the project.

\paragraph{Work Package 1 (WP1):}  This work package concerns the
development of and use of \acsl and \eacsl extensions.


$\bullet$ T1.1: This task will consider needed extensions for handling
real-world embedded systems.  In particular, there will be a focus on
a study of the formal semantics of timed
automaton networks defined in \uppaal and \prism, to determine the
extent to which shared semantics can be assigned making it possible to
carry implementation annotations into such formal models.
%In addition, this task will include initial consultation with engineers from Galois, Inc. to discuss needs for their customers and tools.  Galois is a key player in the space of annotations and tools for critical low-level system verification.

$\bullet$ T1.2: This task will take feedback from applications of
tools to generate tests and proofs (T2) into account, to add annotations
that are focused on heuristic guidance for testing.

One Ph.D. student will conduct this work, which will last for the
entire duration of the project.  Because this aspect is directly tied
to \acsl and \eacsl, and compatibility with Frama-C, we have allocated
money for travel to France towards the end of the year to meet with Frederic Loulergue, a previous
collaborator of the PIs, who has expertise in using Frama-C for 
Internet of Things applications, and full proof automation in Frama-C.

\paragraph{Evaluation:} Evaluation of
WP1 will be determined by ability of embedded engineers to agree that
the key properties, including those related to timed automata models, to be checked are (1) all representable by the
annotations (2) easy to construct (3) easy to read and
(4) maintainable.
% In addition to our own case studies, discussion with Galois engineers will inform our evaluation.

\paragraph{Work Package 2 (WP2):}  This work package covers
automatic translation of \acsl/\eacsl-annotated code 
into a \deepstate test harness (Section~\ref{sec:framac2deepstate}),
development of back-ends for CBMC and SPIN, and
improvements to fuzzers:
\begin{itemize}[labelsep=3pt,leftmargin=12pt]
\item T2.1: This task will optimize the implementation of symbolic
  execution and fuzzing in DeepState, so that \acsl/\eacsl annotations
  and extensions from WP1 can be used effectively.
\item T2.2: This task will develop DeepState back-ends for CBMC and
  SPIN, annotations needed to handle loop bounds,
  memory tracking and matching, and use of feedback from fuzzing.
\end{itemize}

The execution of this work package will also span the entire duration of the project.
Because the tasks in this package are also based on developing
verification and test generation tools (thus formal methods
expertise), the same Ph.D. student will work on WP1 and WP2. 

\paragraph{Evaluation:} Evaluation of
WP2 will be determined by the application
of DeepState harnesses to generate tests for realistic
systems.  We will use benchmarks and simple examples to some
extent, but primarily rely on our connection to case studies.
We will use coverage and faults
detected as the standard evaluation  measures.

\paragraph{Work Package 3 (WP3):}
\textcolor{blue}{%
This work package will focus on consolidating the software developed
in the other work packages in an open-source software tool, supported
by pre-trained models tha assist in code annotation, usable by embedded
software engineers.  These aspects will be evaluated primarily on the case studies described in Section~\ref{sec:case-study}, as a way to inform the methodology and tool developments in the other tasks}.
WP3 includes the following components:
\textcolor{blue}{%
\paragraph{Open-source software tool and annotation assistant.}
Task T3, that develops the open-source software tool, will span the
entire duration of the project, in coordination with the software
development tasks in the other work packages.  All the students in
this project will contribute to task T3.  A core aspect of the tool
will be pre-trained models to 1) produce preliminary annotations for C
code and 2) to complete partial annotations, both as discussed in the
research plan of this proposal.}


\paragraph{Wireless sensor network (WSN) case study on DISCOVER.} This %application
is divided into two tasks:
\noindent  \begin{itemize}[labelsep=3pt,leftmargin=12pt]
\item T4.1: In this task, the %existing SEGA
  wireless sensor node systems will be studied thoroughly to extract the key requirements and characteristics of the embedded system implementations.
  %Timed automaton models of the communication protocol in each system, at different levels of abstraction, may be developed and formally verified in \uppaal and/or \prism, to inform task T1.1.
  The system information and models resulting from this task will inform the semantics design and method developments in WP1 and WP2. 
\item T4.2: This task will apply the tools developed in WP1 and WP2 to the WSN systems, %in order
  to detect and fix bugs in
  the embedded software implementations. % of the
  % communication protocol implementations; in particular, the bugs that cause the intermittent failures in SEGA. % mentioned in Section~\ref{sec:case-study}.
  It will also provide feedback to the other work packages to refine and improve our tools.
  \end{itemize}

\paragraph {Multi-robot system case study on DISCOVER.} This study is divided into three tasks:
\noindent \begin{itemize}[labelsep=3pt,leftmargin=12pt]
\item T5.1: In this task, a standard multi-robot coordination
  algorithm %currently used with our existing multi-robot system
  will be modeled as a network of timed automata.  Using our insights
  into the robotics application, we will express its performance
  specifications, particularly its safety requirements, in temporal
  logics and formally verify or test them in tools like \uppaal,
  \prism, or S-TaLiRo.  This task will extend the developed semantics
  and methods to applications beyond communication protocols, to
  identify further needed runtime extensions and semantic connections
  between timed automata theory, implementation annotations, and
  runtime checks.
\item T5.2: This task will apply the tools developed in WP1 and WP2,
  and the robot simulation environment of the DISCOVER platform, to
  the coordinated multi-robot system, in order to validate the
  implementation code, detect and fix possible bugs, and improve
  the tools developed in this project.
\item T5.3: This task will aim to apply the DeepState-trace driven
  route to produce timed automata skeletons.
  \end{itemize}

As the tasks in this work package are conducted in tandem with WP1 and WP2, to form a feedback loop with the developments in other work packages, it will last for the entire duration of the project.
We expect that groups of undergraduate students, in collaboration with
an embedded systems Ph.D. student and the Ph.D. students in WP1 and WP2, will
perform the work.
Close collaboration with the DISCOVER team, led by Co-PI Nghiem, is expected.

\paragraph{Evaluation:} In essence, this task is the evaluation
aspect of our project, which forms one of the major thrusts of the
project.  The successful application of WP1 and WP2 tools to the case
studies is essentially the driving factor in determining our success
in the project.
%, and the key feedback to drive changes to our research
%priorities or technical choices.
The measure of success is: (1) faults detected and corrected; (2)
functionality proven correct using CBMC, symbolic execution engines,
or SPIN; (3) coverage and other measures of generated tests; and (4)
reported usability and value by engineers,
particularly students.  For T5.3, evaluation will be based on
comparison of extracted skeletons with independently developed full 
models.  Annotation assistance will be evaluated using methods
described in the main proposal (e.g., ability to partially reconstruct
erased annotations).

%\subsubsection{Timeline}
%\label{sec:time-line}


%%% Local Variables:
%%% mode: latex
%%% TeX-master: "main"
%%% End:


% \section{Contributions to Formal Methods and the Field}
% \label{sec:contributions}
% \section{Contributions to Formal Methods and the Field}

The contributions to formal methods proposed include: (1)
Fundamental contributions to integrating formal specification
languages developed for use in static analysis and theorem proving
with dynamic analysis, producing a common semantics for formal,
static, and dynamic checking of correctness; handling of timing and
interrupts are notable examples of problems to be addressed in this effort;
(2) Enhanced ability of fuzzing and other test generation methods to
make use of information from formal specifications, and integrate
feedback about, e.g., specification coverage into test generation
heuristics;
(3) Common semantics and a framework for fuzzing, symbolic execution,
and model checking.
(4) Approaches to using feedback from fuzzing to guide bounded or explicit-state model
checking;
(5) Translations from implementation-level specification to
(probabilistic) timed automata models.
(6) Automated methods for generating and completing code and
verification/testing harness annotations using CodeBERT or other
LLM-based methods, for use across unified formal, static, and dynamic methods.

The contributions to the field include:
(1)  New development and design methods that focus on
implementation-level specification as a guiding
principle for embedded systems; (2) Tactics and strategies that an
also address incorporating the above methods into
legacy efforts, where existing code bases require additional
specification and annotation; (3) Best-practices for combined use of formal, static, and dynamic tools in
more traditional implementation efforts, including ones incorporating
substantial legacy codebases in C.

%\subsubsection{Evaluation Approach}
%\label{sec:eval}

Our \emph{evaluation} of the degree to which these contributions have
been realized is described, integrated with
description of case study efforts, and in more detail in the
collaboration plan document.


%%% Local Variables:
%%% mode: latex
%%% TeX-master: "main"
%%% End:
