\section{Contributions to Formal Methods and the Field}

The contributions to formal methods proposed include: (1)
Fundamental contributions to integrating formal specification
languages developed for use in static analysis and theorem proving
with dynamic analysis, producing a common semantics for formal,
static, and dynamic checking of correctness; handling of timing and
interrupts are notable examples of problems to be addressed in this effort;
(2) Enhanced ability of fuzzing and other test generation methods to
make use of information from formal specifications, and integrate
feedback about, e.g., specification coverage into test generation
heuristics;
(3) Common semantics and a framework for fuzzing, symbolic execution,
and model checking.
(4) Approaches to using feedback from fuzzing to guide bounded or explicit-state model
checking;
(5) Translations from implementation-level specification to
(probabilistic) timed automata models.
(6) Automated methods for generating and completing code and
verification/testing harness annotations using CodeBERT or other
LLM-based methods, for use across unified formal, static, and dynamic methods.

The contributions to the field include:
(1)  New development and design methods that focus on
implementation-level specification as a guiding
principle for embedded systems; (2) Tactics and strategies that an
also address incorporating the above methods into
legacy efforts, where existing code bases require additional
specification and annotation; (3) Best-practices for combined use of formal, static, and dynamic tools in
more traditional implementation efforts, including ones incorporating
substantial legacy codebases in C.

%\subsubsection{Evaluation Approach}
%\label{sec:eval}

Our \emph{evaluation} of the degree to which these contributions have
been realized is described in the work plan above, integrated with
description of case study efforts.