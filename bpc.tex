The goal of the BPC component of this project is to \textit{increase the number of females who are involved or choose careers in computing, at NAU and in the local community of Flagstaff, Arizona.}  Our plan carefully integrates active learning experiences designed for female students at both the undergraduate and middle school/junior high levels.
\textbf{Undergraduate Education Experience -} We will reach female students in two degree programs at the 2nd-year level: Computer Science and Electrical and Computer Engineering. In CS, we will target CS 200 Introduction to Computer Organization; in ECE, we will target EE 215 Internet of Things Design. We will integrate a new project in which teams of female (and possibly male, due to the current lack of females in ECE and CS) students imagine and create exciting and meaningful one-day active learning experiences and projects for female student teams in grades 7-9.  We will provide full support to these teams, especially female students, and design the project so that female students will take leadership roles to gain confidence.  In both courses, we will bring in expert speakers to facilitate development of students’ understanding how to design these projects so they are marker events in the students’ lives. %We will also explicitly address increasing the awareness of the challenges faced by females of all ages in STEM careers.
\textbf{Outreach to grades 7-9 -} As noted above, the undergraduate teams will develop active learning and design project “Build Events” for girls in grades 7-9. We will recruit female undergraduates who have taken CS 200/EE 215 to become mentors in the one-day events for the grade 7-9 students.  We will schedule these events as part of the annual Flagstaff Festival of Science, and plan them for Saturdays to avoid conflicts with school schedules, maximizing participation. The Flagstaff Festival of Science, now in its 34th year and enjoying wide financial and participatory support in the community, holds over 100 events for all ages over a 10-day period in the Fall, and is an ideal venue.
\textbf{Facilities and Support -} By scheduling the grade 7-9 Build Events on Saturdays, we will be able to use the educational laboratories of the School of Informatics, Computing \& Cyber Systems (SICCS) for the Flagstaff Festival of Science events. We have requested \$2,000 for each in years 2 and 3 for materials (primarily embedded development boards) for these experiences.
\textbf{Assessment -} We will conduct focused feedback sessions and administer short surveys of the participants to aid continuous improvement of the activities over the course of the project.