%%%%%%%%%%%%%%%%%%%%%%%%%%%%%%%%%%%%%%%%%%%%%%%%%%%%%%%%%%%%%
\documentclass[12pt]{article}

\usepackage{fullpage}
\input{packages}
% ----------------------------------------
% Frama-C related commands
% ----------------------------------------
\newcommand{\framac}{\textsc{Frama-C}\xspace}
\newcommand{\pathcrawler}{\textsc{Path\-Craw\-ler}\xspace}
\newcommand{\acsl}{\textsc{Acsl}\xspace}
\newcommand{\eacsl}{\textsc{E-Acsl}\xspace}
\newcommand{\conctoseq}{\textsc{conc2seq}\xspace}
\newcommand{\uppaal}{\textsc{Uppaal}\xspace}
\newcommand{\prism}{\textsc{Prism}\xspace}
\newcommand{\spin}{\textsc{Spin}\xspace}
\newcommand{\deepstate}{DeepState\xspace}
\newcommand{\Wp}{\textsc{Wp}\xspace}
\newcommand{\Dafny}{\textsc{Dafny}\xspace}
\newcommand{\Coq}{\textsc{Coq}\xspace}
\newcommand{\Value}{\textsc{Va\-lue}\xspace}
\newcommand{\Eva}{\textsc{Eva}\xspace}
\newcommand{\eva}{\Eva}
% ---------------------------------------- 

\begin{document}

\pagenumbering{gobble}

\begin{center}
{\large\sf\bfseries FMiTF: Track l: Combining Formal, Static, and
  Dynamic Analysis to Verify and Validate Real-World Embedded Systems: Collaboration Plan}
\end{center}

\section{PI Qualifications}

PI Groce has a long history with formal methods, including involvement in design and development of well-known model checkers, and application of model checkers at NASA/JPL on flight software for the Mars rovers, including application of and development contributions to, the SPIN and CBMC model checkers.  Groce was the lead for design and development of automated testing tools and specification languages for the Mars Science Laboratory (Curiosity Rover) project.  He received his PhD under the late Turing laureate Edmund M. Clarke, Jr., as one of the first PhD graduates focusing on model checking for software using SAT and SMT solvers.  More recently, he has primarily focused on developing algorithms and tools for automated software testing; he is a core member of the DeepState and TSTL design and development teams.  Groce is the lead developer for multiple GitHub-hosted testing and verification tools (including DeepState, the focus platform for this project), all with more than 100 GitHub ``stars''; he reports 50-100 bugs detected using such tools to open source projects each year, and has been awarded significant bug bounties for security-critical compiler bugs detected using his work.  Groce was recently chosen by the Bitcoin Core team to investigate the effectiveness of their fuzzing framework.
%Co-PI Loulergue has a long experience in designing parallel programming languages and libraries based on formal semantics. About ten years ago, he started using the \Coq proof assistant in his research on programming language semantics and on the development of parallel programs correct by construction. More recently he has begun a collaboration with colleagues at {\it Commissariat \`a l'\'energie atomique et aux \'energies alternatives} (CEA), extending \framac with new features, motivated by analysis and verification tasks on real-world code. He brings to this project a strong expertise in static analysis and deductive verification of C programs with the \framac framework, and on specification and proof engineering on real-world code. 

Co-PI Nghiem has an extensive background in control and autonomous systems, including terrestrial rovers and unmanned aerial vehicles (UAVs), and application of formal methods in control systems.
He has long experience working with timed automata and the Uppaal tool family.
He developed, and was granted a U.S. patent for, methods for testing and verifying temporal logic specifications of hybrid systems---systems with both continuous and discrete behaviors.
He brings to this project advanced knowledge of real-time embedded systems and practical experience applying verification methods and tools to them.
He will also
apply our methods and approaches to a multi-agent robotics system.


\section{Collaboration Plan}

The PIs are all faculty in the School of Informatics, Computing and Cyber Systems at Northern Arizona University.  They all work in the same building, which also hosts the DISCOVER lab.  The PIs have toured each others' labs and reviewed code developed by the other PIs.  PIs will meet weekly for technical discussion and project status updates, depending on project status (and already do so, in discussions of the preliminary work) to coordinate their efforts, and ensure that tasks in the work plan are proceeding correctly, and to receive feedback on current status of tools.  These meetings will alternate between meetings with just PIs, focused on higher level decision-making, and full meetings focusing on status of individual work packages.
%In addition to weekly project meetings, there will be special out-of-band meetings to demonstrate significant new functionalities in tools, or to focus particularly on application of functionalities to specific DISCOVER components.

The DISCOVER code forms a core concern that helps focus PI interactions, and enables easier communication of technical results between static and dynamic analysis experts and experts in the embedded systems domain.  PIs will also set up a project repository, separate from DeepState or DISCOVER code repositories (already in existence) and use PR, tagged Issues, and other GitHub-supported collaborative software development best practices to ensure documentation of development, and team awareness of code status for components.  GitHub Actions-based will be used to ensure all project members are aware of build or correctness problems with project code.  All project members will also share a Project Slack Instance to facilitate archived, asynchronous discussion of project issues.
PI Groce is well experienced in coordination of complex research efforts focused on open-source tools, as he coordinates efforts for the DeepState project, which has 16 contributors, and has managed releases of the echidna smart contract fuzzer (both tools with more than 700 GitHub stars and with dozens of forks; echidna has more than 20 contributors).

%\subsection{Work Plan and Evaluation Measures}
\label{sec:workplan}
\begin{wrapfigure}[11]{r}{.35\textwidth}
%\begin{figure}[!tp]
%  \centering
  \resizebox{.35\textwidth}{!}{%
  \begin{ganttchart}[%Specs
    hgrid style/.style={black, dotted},
    vgrid, %={*2{black,dotted}, *1{black, dashed},
      %*2{black,dotted}, *1{black, dashed},
      %*2{black,dotted}, *1{black, dashed},
      %*2{black,dotted}, *1{black, solid}},
    x unit=3mm,
    y unit chart=5mm,
    y unit title=5mm,
    %time slot format=isodate,
    title height=1,
    milestone label font=\footnotesize,
    group label font=\bfseries\footnotesize,
    title label font=\bfseries\footnotesize,
    link/.style={->, thick},
    %bar/.style={fill=blue},
    %bar height=0.7,
    %group right shift=0,
    %group top shift=0.7,
    %group height=.3,
    %group peaks width={0.2},
    %inline
    ]{1}{36}
    % labels
    % \gantttitle{A two-years project}{24}\\  % title 1 
    \gantttitle[]{Year 1}{12}                 % title 1
    \gantttitle[]{Year 2}{12}
    \gantttitle[]{Year 3}{12} \\
    \gantttitle{Q1}{3}                      % title 3
    \gantttitle{Q2}{3}
    \gantttitle{Q3}{3}
    \gantttitle{Q4}{3}
    \gantttitle{Q1}{3}
    \gantttitle{Q2}{3}
    \gantttitle{Q3}{3}
    \gantttitle{Q4}{3}
    \gantttitle{Q1}{3}
    \gantttitle{Q2}{3}
    \gantttitle{Q3}{3} 
    \gantttitle{Q4}{3}\\    

    % \ganttgroup[inline=false]{Group 1}{1}{5}\\ 
    % \ganttbar[progress=10,inline=false]{Planning}{1}{4}\\
    % \ganttmilestone[inline=false]{Milestone 1}{9} \\

    % \ganttgroup[inline=false]{Group 2}{6}{12} \\ 
    % \ganttbar[progress=2,inline=false]{test1}{10}{19} \\
    % \ganttmilestone[inline=false]{Milestone 2}{17} \\
    % \ganttbar[progress=5,inline=false]{test2}{11}{20} \\
    % \ganttmilestone[inline=false]{Milestone 3}{22} \\       

    % \ganttgroup[inline=false]{Group 3}{13}{24} \\ 
    % \ganttbar[progress=90,inline=false]{Task A}{13}{15} \\ 
    % \ganttbar[progress=50,inline=false, bar progress label node/.append style={below left= 10pt and 7pt}]{Task B}{13}{24} \\ \\
    % \ganttbar[progress=30,inline=false]{Task C}{15}{16}\\ 
    % \ganttbar[progress=70,inline=false]{Task D}{18}{20} \\ 

    \ganttgroup[
        group/.append style={fill=blue}
    ]{WP1}{1}{36}\\ [grid]
    \ganttbluebar[
        name=T11
    ]{T1.1}{1}{12}\\ [grid]
    \ganttbluebar[
        name=T12
    ]{T1.2}{13}{36}\\ [grid]
    % \ganttlinkedbluebar{}{2014-10-7}{2014-10-10}
    % \ganttlinkedbluebar{}{2014-10-14}{2014-10-15}
    % \ganttlinkedbluebar{}{2014-10-17}{2014-10-17}
    % \ganttlinkedbluebar[name=FMEend]{}{2014-10-21}{2014-10-24}
    % \ganttlinkedbluebar{}{2014-10-28}{2014-10-31}\\ [grid]
    % \ganttbluebar[name=Manual]{Manual}{2014-10-30}{2014-10-31}
    % \ganttlinkedbluebar{}{2014-11-4}{2014-11-7} \ganttnewline[thick, black]

    \ganttgroup[
        group/.append style={fill=blue}
    ]{WP2}{1}{36}\\ [grid]
    \ganttbluebar[
        name=T21
    ]{T2.1}{1}{12}\\ [grid]
    \ganttbluebar[
        name=T22
    ]{T2.2}{13}{36}\\ [grid]

    \ganttgroup[
        group/.append style={fill=blue}
    ]{WP3}{1}{36}\\ [grid]
    \ganttbluebar[
        name=T3
    ]{T3}{1}{36}\\ [grid]
    \ganttbluebar[
        name=T41
    ]{T4.1}{1}{12}\\ [grid]
    \ganttbluebar[
        name=T42
    ]{T4.2}{13}{36}\\ [grid]
    \ganttbluebar[
        name=T51
    ]{T5.1}{7}{12}\\ [grid]        
    \ganttbluebar[
        name=T52
    ]{T5.2}{13}{36}\\ [grid]
    \ganttbluebar[
        name=T53
    ]{T5.3}{25}{36}    
    % %Implementing links
    % \ganttlink[link mid=0.75]{Documentation}{FME}
    % \ganttlink{FMETutorial}{FME}
  \end{ganttchart}}%
\caption{Project schedule.}%
\label{fig:project-schedule}%
% \end{figure}
%\vspace{-0.4in}
\end{wrapfigure}

The project will be organized into two phases, described by work
packages.  In the first phase, T4.1 will be conducted along with and inform T1.1 (see Figure~\ref{fig:project-schedule}).
In the second phase, the focus will be on the application of tools in T1.2 in tandem with T2.
Tasks related to the case studies %(tasks T4.1, and T4.2)
will help refine the developed tools especially in the final phases of the project.

\paragraph{Work Package 1 (WP1):}  This work package concerns the
development of and use of \acsl and \eacsl extensions.


$\bullet$ T1.1: This task will consider needed extensions for handling
real-world embedded systems.  In particular, there will be a focus on
a study of the formal semantics of timed
automaton networks defined in \uppaal and \prism, to determine the
extent to which shared semantics can be assigned making it possible to
carry implementation annotations into such formal models.
%In addition, this task will include initial consultation with engineers from Galois, Inc. to discuss needs for their customers and tools.  Galois is a key player in the space of annotations and tools for critical low-level system verification.

$\bullet$ T1.2: This task will take feedback from applications of
tools to generate tests and proofs (T2) into account, to add annotations
that are focused on heuristic guidance for testing.

One Ph.D. student will conduct this work, which will last for the
entire duration of the project.  Because this aspect is directly tied
to \acsl and \eacsl, and compatibility with Frama-C, we have allocated
money for travel to France towards the end of the year to meet with Frederic Loulergue, a previous
collaborator of the PIs, who has expertise in using Frama-C for 
Internet of Things applications, and full proof automation in Frama-C.

\paragraph{Evaluation:} Evaluation of
WP1 will be determined by ability of embedded engineers to agree that
the key properties, including those related to timed automata models, to be checked are (1) all representable by the
annotations (2) easy to construct (3) easy to read and
(4) maintainable.
% In addition to our own case studies, discussion with Galois engineers will inform our evaluation.

\paragraph{Work Package 2 (WP2):}  This work package covers
automatic translation of \acsl/\eacsl-annotated code 
into a \deepstate test harness,
development of back-ends for CBMC and SPIN, and
improvements to fuzzers:
\begin{itemize}[labelsep=3pt,leftmargin=12pt]
\item T2.1: This task will optimize the implementation of symbolic
  execution and fuzzing in DeepState, so that \acsl/\eacsl annotations
  and extensions from WP1 can be used effectively.
\item T2.2: This task will develop DeepState back-ends for CBMC and
  SPIN, annotations needed to handle loop bounds,
  memory tracking and matching, and use of feedback from fuzzing.
\end{itemize}

The execution of this work package will also span the entire duration of the project.
Because the tasks in this package are also based on developing
verification and test generation tools (thus formal methods
expertise), the same Ph.D. student will work on WP1 and WP2. 

\paragraph{Evaluation:} Evaluation of
WP2 will be determined by the application
of DeepState harnesses to generate tests for realistic
systems.  We will use benchmarks and simple examples to some
extent, but primarily rely on our connection to case studies.
We will use coverage and faults
detected as the standard evaluation  measures.

\paragraph{Work Package 3 (WP3):}
This work package will focus on consolidating the software developed
in the other work packages in an open-source software tool, supported
by pre-trained models that assist in code annotation, usable by embedded
software engineers.  These aspects will be evaluated primarily on the
case studies described in the project description, as a way to inform the methodology and tool developments in the other tasks.
WP3 includes the following components:
\paragraph{Open-source software tool and annotation assistant.}
Task T3, that develops the open-source software tool, will span the
entire duration of the project, in coordination with the software
development tasks in the other work packages.  All the students in
this project will contribute to task T3.  A core aspect of the tool
will be pre-trained models to 1) produce preliminary annotations for C
code and 2) complete partial annotations, both as discussed in the
research plan of this proposal.


\paragraph{Wireless sensor network (WSN) case study on DISCOVER.} This %application
is divided into two tasks:
\noindent  \begin{itemize}[labelsep=3pt,leftmargin=12pt]
\item T4.1: In this task, the %existing SEGA
  wireless sensor node systems will be studied thoroughly to extract the key requirements and characteristics of the embedded system implementations.
  %Timed automaton models of the communication protocol in each system, at different levels of abstraction, may be developed and formally verified in \uppaal and/or \prism, to inform task T1.1.
  The system information and models resulting from this task will inform the semantics design and method developments in WP1 and WP2. 
\item T4.2: This task will apply the tools developed in WP1 and WP2 to the WSN systems, %in order
  to detect and fix bugs in
  the embedded software implementations. % of the
  % communication protocol implementations; in particular, the bugs that cause the intermittent failures in SEGA. % mentioned in Section~\ref{sec:case-study}.
  It will also provide feedback to the other work packages to refine and improve our tools.
  \end{itemize}

\paragraph {Multi-robot system case study on DISCOVER.} This study is divided into three tasks:
\noindent \begin{itemize}[labelsep=3pt,leftmargin=12pt]
\item T5.1: In this task, a standard multi-robot coordination
  algorithm %currently used with our existing multi-robot system
  will be modeled as a network of timed automata.  Using our insights
  into the robotics application, we will express its performance
  specifications, particularly its safety requirements, in temporal
  logics and formally verify or test them in tools like \uppaal,
  \prism, or S-TaLiRo.  This task will extend the developed semantics
  and methods to applications beyond communication protocols, to
  identify further needed runtime extensions and semantic connections
  between timed automata theory, implementation annotations, and
  runtime checks.
\item T5.2: This task will apply the tools developed in WP1 and WP2,
  and the robot simulation environment of the DISCOVER platform, to
  the coordinated multi-robot system, in order to validate the
  implementation code, detect and fix possible bugs, and improve
  the tools developed in this project.
\item T5.3: This task will aim to apply the DeepState-trace driven
  route to produce timed automata skeletons.
  \end{itemize}

As the tasks in this work package are conducted in tandem with WP1 and WP2, to form a feedback loop with the developments in other work packages, it will last for the entire duration of the project.
We expect that groups of undergraduate students, in collaboration with
an embedded systems Ph.D. student and the Ph.D. students in WP1 and WP2, will
perform the work.
Close collaboration with the DISCOVER team, led by Co-PI Nghiem, is expected.

\paragraph{Evaluation:} In essence, this task is the evaluation
aspect of our project, which forms one of the major thrusts of the
project.  The successful application of WP1 and WP2 tools to the case
studies is essentially the driving factor in determining our success
in the project.
%, and the key feedback to drive changes to our research
%priorities or technical choices.
The measure of success is: (1) faults detected and corrected; (2)
functionality proven correct using CBMC, symbolic execution engines,
or SPIN; (3) coverage and other measures of generated tests; and (4)
reported usability and value by engineers,
particularly students.  For T5.3, evaluation will be based on
comparison of extracted skeletons with independently developed full 
models.  Annotation assistance will be evaluated using methods
described in the main proposal (e.g., ability to partially reconstruct
erased annotations).


%\subsection{Risks and Mitigations}

%Based on the project goals, we view the primary risk as loss of focus on a coherent application to the case study problems.  This project is to be driven by embedded systems engineering needs, since we view advances that enable real-world usability and cost effectiveness as they key missing features in the application of formal methods to embedded systems engineering.  

%The second primary risk addressed by the plan is that while many research efforts do not require production-quality software, we aim to primarily operate through enhancements to existing, production-quality, tools.  As noted, we will be applying a more rigorous development process than is frequently used in academic software systems.  Because advances in formal methods application to real software are frequently rooted in heuristic solutions (since the general problem is usually infeasible, due to, e.g., Rice's theorem), correct and robust tool implementation work is essential to proper evaluation of advances.
\end{document}
